%%%%%%%%%%%%%%%%%%%%%%%%%%%%%%%%%%%%%%%%%
% baposter Portrait Poster
% LaTeX Template
% Version 1.0 (15/5/13)
%
% Created by:
% Brian Amberg (baposter@brian-amberg.de)
%
% This template has been downloaded from:
% http://www.LaTeXTemplates.com
%
% License:
% CC BY-NC-SA 3.0 (http://creativecommons.org/licenses/by-nc-sa/3.0/)
%
%%%%%%%%%%%%%%%%%%%%%%%%%%%%%%%%%%%%%%%%%

%----------------------------------------------------------------------------------------
%	PACKAGES AND OTHER DOCUMENT CONFIGURATIONS
%----------------------------------------------------------------------------------------

\documentclass[a0paper,portrait]{baposter}

\usepackage{tabulary}
\usepackage[font=small,labelfont=bf]{caption} % Required for specifying captions to tables and figures
\usepackage{booktabs} % Horizontal rules in tables
\usepackage{relsize} % Used for making text smaller in some places
\usepackage[urlcolor  = blue]{hyperref}


\definecolor{bordercol}{RGB}{5,2,70} % Border color of content boxes
\definecolor{headercol1}{RGB}{5,2,70} % Background color for the header in the content boxes (left side)
\definecolor{headercol2}{RGB}{5,2,70} % Background color for the header in the content boxes (right side)
\definecolor{headerfontcol}{RGB}{255,255,255} % Text color for the header text in the content boxes
\definecolor{boxcolor}{RGB}{255,255,255} % Background color for the content in the content boxes

\begin{document}

\background{ % Set the background to an image (background.pdf)

}

\begin{poster}{
grid=false,
borderColor=bordercol, % Border color of content boxes
headerColorOne=headercol1, % Background color for the header in the content boxes (left side)
headerColorTwo=headercol1, % Background color for the header in the content boxes (right side)
headerFontColor=headerfontcol, % Text color for the header text in the content boxes
boxColorOne=boxcolor, % Background color for the content in the content boxes
headershape=roundedright, % Specify the rounded corner in the content box headers
headerfont=\Large\sf\bf, % Font modifiers for the text in the content box headers
textborder=rectangle,
background=user,
headerborder=open, % Change to closed for a line under the content box headers
boxshade=plain
}
{}
%
%----------------------------------------------------------------------------------------
%	TITLE AND AUTHOR NAME
%----------------------------------------------------------------------------------------
%
%\vspace{2em}
{\sf\bf Analysis { } of { } Discourse { }  Macrostructure  { } in { } Schizophrenia: a { } Corpus { }  Study} % Poster title
{\vspace{1em} Galina Ryazanskaya, Khudyakova Mariya\\ % Author names
{\smaller Fundamental and Applied Linguistics, HSE, Moscow}\\% Author email addresses
{..}\\
}
%

%----------------------------------------------------------------------------------------
%	INTRODUCTION
%----------------------------------------------------------------------------------------

\headerbox{Introduction}{name=introduction,column=0,row=0, span=3}{
\begin{itemize}
\item Schizophrenia is a complex highly heritable mental disorder, characterized by a disintegration of the process of thinking, loss of contact with reality, and emotional responsiveness 
\item One of the diagnostic criteria for schizophrenia in both DSM-5 \cite{Tandon2013} and ICD-10 \cite{ICD-10} is disorganized, bizarre speech  
\item There are several measures of discourse incoherence, which are well studied in aphatic speech \cite{KhudyakovaPreprint}
\item We concentrated our attention on measures from several different articles: 
\begin{itemize}
    \item 5-score scale for local and global coherence \cite{methodfrom}
    \item violations of completeness (measure of informativeness) \cite{Christiansen1995, KhudyakovaPreprint}
\end{itemize} 
\end{itemize}

}

%\headerbox{True vs False Motifs}{name=introduction2,column=1,row=0, span=2}{
%\includegraphics[scale=0.42]{motif_study}
%}

%----------------------------------------------------------------------------------------
%	MATERIALS AND METHODS
%----------------------------------------------------------------------------------------

\headerbox{Materials and Methods}{name=methods,column=0,below=introduction, span=3}{


\underline{\textbf{Hypothesis}}
(based on previous research \cite{Ditman2010})
\begin{itemize}
\item Lower scores on local and global coherence measures for participants with schizophrenia
\item Somewhat lower scores on informativeness measures for participants with schizophrenia due to executive dysfunction
\end{itemize}


\underline{\textbf{Participants}}

\begin{itemize}
\item 9 outpatients diagnosed with schizophrenia and 10 controls from Russian CliPS \cite{RussianСliPS}
\end{itemize}

\underline{\textbf{Method}}

\begin{itemize}
\item The Pear Film \cite{Chafe1980} retelling
\item ELAN annotation as in Russian CliPS \cite{RussianСliPS}
\item Utterance segmentation,  application of coherence scaling and violations of completeness measuring
\end{itemize}

}

%----------------------------------------------------------------------------------------
%	CONCLUSIONS
%----------------------------------------------------------------------------------------
\headerbox{Conclusions}{name=conclusion,span=1,column=2,below=methods}{
Absence of significant difference may be due to:
\begin{itemize}
\item Small sample size
\item Differences in symptoms
\item Treatment
\item Subjectivity of some methods (+absence of second annotator)
\item Inapplicability of the methods
\end{itemize}
}

%----------------------------------------------------------------------------------------
%	RESULTS
%----------------------------------------------------------------------------------------

\headerbox{Results}{name=results2,span=2,column=0,below=methods, bottomaligned=conclusion}{ 
There were no statistically significant differences between two groups\\
The results of t-test (and one Kruskal-Wallis test):
\begin{center}
Characteristics of the texts and differences between groups
\begin{tabular}{ c c c c c c }
 \hline
  & \multicolumn{2}{c}{\underline{Control}} & \multicolumn{2}{c}{\underline{Schizo}} & \\
 measure & Mean & SD & Mean & SD & p-value\\ 
 \hline
 Global coherence & 4.39 & 0.39 & 4.10 & 0.53 & 0.201\\ 
 Local coherence & 3.77 & 0.26 & 3.45 & 0.44 & 0.120\\
 Absolute completeness violations & 0.50 & 0.70 & 1.44 & 1.42 & 0.098\\ 
 Average completeness violations & 0.02 & 0.02 & 0.05 & 0.14 & 0.098\\
 Number of utterances & 38.40 & 13.29 & 27.11 & 12.48 & 0.073\\ 
 \hline
\end{tabular}
\end{center}
}

%----------------------------------------------------------------------------------------
%	REFERENCES
%----------------------------------------------------------------------------------------

\headerbox{References}{name=references,column=0, row=0.75, span=3}{

\scriptsize % Reduce the font size in this block
\renewcommand{\section}[2]{\vskip 0.05em} % Get rid of the default "References" section title
\nocite{*} % Insert publications even if they are not cited in the poster

\bibliographystyle{apalike}
\bibliography{abstract} % Use sample.bib as the bibliography file
}

%----------------------------------------------------------------------------------------

\end{poster}
\end{document}